% Options for packages loaded elsewhere
\PassOptionsToPackage{unicode}{hyperref}
\PassOptionsToPackage{hyphens}{url}
%
\documentclass[
]{article}
\usepackage{amsmath,amssymb}
\usepackage{lmodern}
\usepackage{ifxetex,ifluatex}
\ifnum 0\ifxetex 1\fi\ifluatex 1\fi=0 % if pdftex
  \usepackage[T1]{fontenc}
  \usepackage[utf8]{inputenc}
  \usepackage{textcomp} % provide euro and other symbols
\else % if luatex or xetex
  \usepackage{unicode-math}
  \defaultfontfeatures{Scale=MatchLowercase}
  \defaultfontfeatures[\rmfamily]{Ligatures=TeX,Scale=1}
\fi
% Use upquote if available, for straight quotes in verbatim environments
\IfFileExists{upquote.sty}{\usepackage{upquote}}{}
\IfFileExists{microtype.sty}{% use microtype if available
  \usepackage[]{microtype}
  \UseMicrotypeSet[protrusion]{basicmath} % disable protrusion for tt fonts
}{}
\makeatletter
\@ifundefined{KOMAClassName}{% if non-KOMA class
  \IfFileExists{parskip.sty}{%
    \usepackage{parskip}
  }{% else
    \setlength{\parindent}{0pt}
    \setlength{\parskip}{6pt plus 2pt minus 1pt}}
}{% if KOMA class
  \KOMAoptions{parskip=half}}
\makeatother
\usepackage{xcolor}
\IfFileExists{xurl.sty}{\usepackage{xurl}}{} % add URL line breaks if available
\IfFileExists{bookmark.sty}{\usepackage{bookmark}}{\usepackage{hyperref}}
\hypersetup{
  pdftitle={Week 4 Linux Notes},
  pdfauthor={Freddie Grogono},
  hidelinks,
  pdfcreator={LaTeX via pandoc}}
\urlstyle{same} % disable monospaced font for URLs
\usepackage[margin=1in]{geometry}
\usepackage{graphicx}
\makeatletter
\def\maxwidth{\ifdim\Gin@nat@width>\linewidth\linewidth\else\Gin@nat@width\fi}
\def\maxheight{\ifdim\Gin@nat@height>\textheight\textheight\else\Gin@nat@height\fi}
\makeatother
% Scale images if necessary, so that they will not overflow the page
% margins by default, and it is still possible to overwrite the defaults
% using explicit options in \includegraphics[width, height, ...]{}
\setkeys{Gin}{width=\maxwidth,height=\maxheight,keepaspectratio}
% Set default figure placement to htbp
\makeatletter
\def\fps@figure{htbp}
\makeatother
\setlength{\emergencystretch}{3em} % prevent overfull lines
\providecommand{\tightlist}{%
  \setlength{\itemsep}{0pt}\setlength{\parskip}{0pt}}
\setcounter{secnumdepth}{-\maxdimen} % remove section numbering
\ifluatex
  \usepackage{selnolig}  % disable illegal ligatures
\fi

\title{Week 4 Linux Notes}
\author{Freddie Grogono}
\date{21/10/2021}

\begin{document}
\maketitle

\hypertarget{login}{%
\section{Login}\label{login}}

To login into Linux do:

Username: **
\href{mailto:fg17761@bc4login.acrc.bris.ac.uk}{\nolinkurl{fg17761@bc4login.acrc.bris.ac.uk}}
\textbf{ Then: } university password **

\hypertarget{directories}{%
\section{Directories}\label{directories}}

\begin{itemize}
\tightlist
\item
  Pathnames and (/)

  \begin{itemize}
  \tightlist
  \item
    divider for the path.
  \end{itemize}
\item
  The Directories . and ..
\end{itemize}

Current and previous disrectories. - Home directories (\textasciitilde)
- Tilde represents a user's home directory.

Finding where you are (pwd): - Command to print working directory

Changing to a different Directory (cd) - Command to change directory

New Files and Directories - Listing files and directories (ls) - Making
Directories (mkdir)

Tip: - using cd \textasciitilde{} will take you to your home directory
in Linux - You can use clear or ctrl+l in terminal to clear the terminal
screen to make it easy to follow

Make a new directory in the intro\_to\_linux directory

Change into that directory:
{[}\href{mailto:fg177616@bc4login1}{\nolinkurl{fg177616@bc4login1}}
\textasciitilde{]}\$ cd intro\_to\_linux/

Check whats in there:
{[}\href{mailto:fg17761@bc4login1}{\nolinkurl{fg17761@bc4login1}}
intro\_to\_linux{]}\$ ls

authors example.txt

Create the new directory and check it's there:
{[}\href{mailto:fg17761@bc4login1}{\nolinkurl{fg17761@bc4login1}}
intro\_to\_linux{]}\$ mkdir create
{[}\href{mailto:fg17761@bc4login1}{\nolinkurl{fg17761@bc4login1}}
intro\_to\_linux{]}\$ ls

authors create example.txt

File Manipulation: - Copying Files (cp) - Moving Files (mv) - Removing
Files and directories (rm)

Warning: - Using rm cannot be undone, it is a permanent delete.

Examining File Contents: - Displaying the contents of a file on the
screen (cat, less) - The first lines in a file (head) - The last lines
in a file (tail) - Searching the contents of a file (grep) - Counting
with grep (-c) - UNIX is case sensitive - grep -i

Tip: u to page up, space to page down, q to quit ctrl + c This shortcut
will abort the execution of a program. Use it to get back to the
terminal, if you get stuck using a command.

Filenames Wildcards

? Any one character

e.g.~ab? will match aba, abb, ab1, ab2 and so on, for all letters and
numbers (a-z, 0-9), but won't match ab

\begin{itemize}
\tightlist
\item
  Zero or more characters.
\end{itemize}

e.g.~ab* will match ab, abb, abbbbb, abc, abbbcd1, etc

Filename Conventions (e.g.~.c)

rm and * could delete everything.

e.g. rm \emph{.txt (will permanently delete all files with extension
.txt in the current directory.) Make sure that there is NO space between
the star } and the dot . rm * .txt (will permanently delete ALL files in
current directory.)

Getting Help man :to display the page manual man -k :displays a list of
topics in pages related to the query Google (other search engines are
available)

Tip Remember q to quit, / to search a pattern. In addition to man
command, commands usually print help using flags -h and/or --help
e.g.~command -h and/or command --help

Note Not every command has a manual page or support for help flags. They
may support, all, some or none of those options.

Maniputaing Files: - sort sorts the lines of a text file line by line -
uniq searches for and removes duplicate lines in a file

Tip - uniq considers an entry to be duplicated, only if they are in
adjacent lines. - sort and uniq are often used in combination

Redirection: Redirecting the Output \textgreater{} (overwrites)
\textgreater\textgreater{} (appends) Redirecting the Input \textless{}

Warning: Overwriting a file will destroy its content. e.g.~command
\textgreater{} filename.ext If file filename.ext does not exist, it will
be created. If it exists, its content will be replaced by the output of
command

Pipes and Pipelines: Take the standard output of one command and feed it
in to the standard input of the next Uses the pipe (vertical bar) symbol
`\textbar{}' No intermediate files! Efficient

Text Editors: Not Word vim is mentioned a lot but don't use just yet We
recommend nano \^{}O to save \^{}X to exit

Commands: ssh = secure shell pwd = print working directory cd = current
directory ls = list mkdir = make directory

\end{document}
